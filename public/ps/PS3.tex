%2multibyte Version: 5.50.0.2960 CodePage: 1251
%\renewcommand{\baselinestretch}{1.8}
%\input{tcilatex}
%\input{tcilatex}
%\usepackage{harvard}
%\input{tcilatex}
%\input{tcilatex}


\documentclass[12pt]{article}
%%%%%%%%%%%%%%%%%%%%%%%%%%%%%%%%%%%%%%%%%%%%%%%%%%%%%%%%%%%%%%%%%%%%%%%%%%%%%%%%%%%%%%%%%%%%%%%%%%%%%%%%%%%%%%%%%%%%%%%%%%%%%%%%%%%%%%%%%%%%%%%%%%%%%%%%%%%%%%%%%%%%%%%%%%%%%%%%%%%%%%%%%%%%%%%%%%%%%%%%%%%%%%%%%%%%%%%%%%%%%%%%%%%%%%%%%%%%%%%%%%%%%%%%%%%
\usepackage{amsfonts}
\usepackage{amssymb}
\usepackage{amsmath}
\usepackage{graphicx}
\usepackage{amsthm}
\usepackage{framed}
\usepackage{times}
\usepackage{mathptmx}
\usepackage{clrscode}
\usepackage{natbib}
\usepackage{booktabs,caption,fixltx2e}
\usepackage[flushleft]{threeparttable}
\usepackage{xpatch}
\usepackage{tabu}
\usepackage{setspace}
\usepackage[margin=1in]{geometry}

\setcounter{MaxMatrixCols}{10}
%TCIDATA{OutputFilter=LATEX.DLL}
%TCIDATA{Version=5.50.0.2960}
%TCIDATA{Codepage=1251}
%TCIDATA{<META NAME="SaveForMode" CONTENT="1">}
%TCIDATA{BibliographyScheme=Manual}
%TCIDATA{Created=Friday, July 11, 2003 08:16:50}
%TCIDATA{LastRevised=Tuesday, July 22, 2014 01:00:36}
%TCIDATA{<META NAME="GraphicsSave" CONTENT="32">}
%TCIDATA{Language=American English}

\makeatletter
\chardef\TPT@@@asteriskcatcode=\catcode`*
\catcode`*=11
\xpatchcmd{\threeparttable}
{\TPT@hookin{tabular}}
{\TPT@hookin{tabular}\TPT@hookin{tabu}}
{}{}
\catcode`*=\TPT@@@asteriskcatcode
\makeatother
\parskip 0cm
\parindent .4cm
\textwidth=6.5in
\oddsidemargin .0in
\topmargin -.5in
\textheight 9in
\doublespacing
\newtheorem{proposition}{Proposition}
\newtheorem{conjecture}{Conjecture}
\newtheorem{corollary}{Corollary}
\newtheorem{lemma}{Lemma}
\theoremstyle{definition}
\newtheorem{definition}{Definition}
\newtheorem{assumption}{Assumption}
\newtheorem{example}{Example}
\newtheorem{am}{Alternative Model}
\renewcommand{\proofname}{\bf Proof}
\renewcommand{\qedsymbol}{$\blacksquare$}

\newcommand{\bpi}{\boldsymbol{\pi}}
\newcommand{\he}{\hat{\eta}}
\newcommand{\etahat}{\hat{\eta}}
\newcommand{\hx}{\hat{x}}
\newcommand{\that}{\hat{\theta}}
\newcommand{\hv}{\hat{\theta}}
\newcommand{\hvl}{\hv_L}
\newcommand{\hvh}{\hv_H}
\newcommand{\vl}{\theta_L}
\newcommand{\vh}{\theta_H}
\newcommand{\hp}{\hat{p}}
%\newcommand{\bv}{\bar{v}}
\newcommand{\bv}{\theta}
\newcommand{\s}{\sigma}
\newcommand{\E}{\mathbb{E}}
\newcommand{\ind}{\mathbbm{1}}
\newcommand{\bigmid}{\;\bigg|\;}
\newcommand{\lbigmid}{\;\big|\;}
\newcommand{\cas}{\overset{\textnormal{a.s.}}{\longrightarrow}} 


\newcommand{\al}{\alpha}


%from P and A file

\newcommand{\hy}{\hat{y}}
\newcommand{\hh}{\hat{h}}
\newcommand{\hI}{\hat{I}}
\newcommand{\hF}{\hat{F}}
\newcommand{\hE}{\hat{E}}

\newcommand{\ah}{\hat{\theta}}
\newcommand{\at}{\tilde{\theta}}
\newcommand{\Fh}{\widehat{F}}
\newcommand{\eh}{\hat{e}}
\newcommand{\yh}{\hat{y}}
%\newcommand{\sh}{\widehat{\sigma}}
\newcommand{\sh}{\sigma}

\newcommand{\et}{\tilde{e}}
\newcommand{\zt}{\tilde{z}}
\newcommand{\Zt}{\widetilde{Z}}
\newcommand{\es}{e^*}
\newcommand{\xh}{\hat{x}}
\newcommand{\zh}{\hat{z}}
\newcommand{\Zh}{\widehat{Z}}
\newcommand{\e}{\epsilon}
\newcommand{\D}{\partial}
\newcommand{\Eh}{\widehat{\mathbb{E}}}


\raggedbottom
\begin{document}
	
	
	\begin{center}
		\underline{{\Large Problem Set Vincent}}
		
		\textbf{[Due at the beginning of class on Thursday, March 16th]}
	\end{center}
\bigskip

\underline{Question 1}:

Consider a simple two-period model of intertemporal choice. Suppose that a
person receives income \$37,254 in period 1 and additional income \$33,169
in period 2. The market interest rate at which the person can both borrow
and save is 3\%. Finally, the person's preferences are given by 
\[
U(c_{1},c_{2})=\frac{4}{3}\left( c_{1}\right) ^{3/4}\text{ }+\text{ }\delta 
\frac{4}{3}\left( c_{2}\right) ^{3/4}. 
\]

\textbf{(a)} Derive the budget constraint that the person faces.

\textbf{(b)} Solve for the optimal $c_{1}$ and $c_{2}$ as a function of $%
\delta $.

\textbf{(c)} As $\delta $ increases, what happens to the optimal $c_{1}$ and 
$c_{2}$? Provide some intuition for your answer.

\textbf{(d)} For what values of $\delta $ will the person save, and for what
values of $\delta $ will she borrow?\newline
[\textbf{Note: Please report your answer to 5 decimal points.]}

\textbf{(e)} For what values of $\delta $ is the person's $c_{1}$ larger
than her $c_{2}$? How does this compare to the condition that we discussed
in class (for the case of log utility)?

\textbf{(f)} Consider the following alternative income streams:

\begin{quotation}
	Alternative A: Receive \$47,923 in period 1 and \$21,535 in period 2.
	
	Alternative B: Receive \$28,716 in period 1 and \$41,963 in period 2.
\end{quotation}

Discuss how the person's optimal consumption path under these alternatives
would compare to her optimal consumption path under the initial income
stream --- that is, discuss whether $c_{1}$ is larger, smaller, or the same,
and discuss whether $c_{2}$ is larger, smaller, or the same. In addition,
discuss how the person's period-1 saving under these alternatives would
compare to her priod-1 saving under the initial income stream. \textbf{%
	[Note: You can and should answer part (f) without re-solving for the
	person's behavior.]}

\medskip

\underline{Question 2 [Optional for two bonus points]}:

Consider a model exactly like that in Question 1 --- where the person
receives income \$37,254 in period 1 and additional income \$33,169 in
period 2 --- except let's now suppose that the person faces a liquidity
constraint. Specifically, she can still save at an interest rate of 3\%, but
if she borrows, then she must pay an interest rate of 9\%.

\textbf{(a)} If the person wants to save, the relevant interest rate is 3\%.
For what values of $\delta $ is it optimal to save? [Hint: You already know
the answer from Question 1.]

\textbf{(b)} If the person wants to borrow, the relevant interest rate is
9\%.

\begin{quotation}
	(i) Suppose the interest rate is 9\%, and solve for the optimal $c_{1}$ and $%
	c_{2}$ \newline
	as a function of $\delta $.
	
	(ii) If the interest rate is 9\%, for what values of $\delta $ is it optimal
	to borrow?\newline
	[\textbf{Note: Please report your answer to 5 decimal points.]}
\end{quotation}

\textbf{(c)} Given the liquidity constraint, for what values of $\delta $ is
it optimal to neither borrow nor save?\newline
[Hint: Two conditions must hold: (i) $\delta $ must be such that the person
does NOT want to save at an interest rate of 3\%, and (ii) $\delta $ must be
such that the person does NOT want to borrow at an interest rate of 9\%.]

\textbf{(d)} Draw three pictures that illustrate the three cases --- when
it's optimal to save, when it's optimal to borrow, and when it's optimal to
neither borrow nor save. Each picture should depict (i) the budget
constraint, (ii) some indifference curves, and (iii) the optimal $c_{1}$ and 
$c_{2}$.

\underline{Question 3}

Suppose that Mr. Z has a tree growing in his yard that has become too large,
and he must cut it down within the next four years. Cutting down the tree
requires effort, and he would like to wait and cut down the tree later.
However, each year that he delays, the tree gets bigger and requires even
more effort to cut down. Specifically, suppose that the effort cost to cut
down the tree in year $\bar{\tau}$ is $c(\bar{\tau})$, where $c(1)=13$, $%
c(2)=20$, $c(3)=32$, and $c(4)=55$. Suppose that Mr. Z is an exponential
discounter with discount factor $\delta $, and that the only utility
consequence of cutting down the tree is the disutility of effort experienced
when the tree is cut.

\textbf{(a)} From a year-1 perspective:

\begin{quotation}
	(i) For what $\delta $ does Mr. Z prefer cutting in year 1 rather than year
	2?\newline
	(ii) For what $\delta $ does Mr. Z prefer cutting in year 2 rather than year
	3?\newline
	(iii) For what $\delta $ does Mr. Z prefer cutting in year 3 rather than
	year 4?
\end{quotation}

\textbf{(b)} As a function of $\delta $, when is Mr. Z's preferred time to
cut down the tree?

\textbf{(c)} Suppose Mr. Z does not harvest the tree in year 1. From a
year-2 perspective:

\begin{quotation}
	(i) For what $\delta $ does Mr. Z prefer cutting in year 2 rather than year
	3?\newline
	(ii) For what $\delta $ does Mr. Z prefer cutting in year 3 rather than year
	4?
\end{quotation}

\textbf{(d)} Is there any $\delta $ for which Mr. Z would decide in year 2
to deviate from his initial year-1 plan?

\bigskip

\underline{Question 4}

We can write the standard $T$-period saving-consumption model as:

\begin{quotation}
	Choose $(c_{1},c_{2},...,c_{T})$ to maximize%
	\[
	U(c_{1},c_{2},...,c_{T})=u(c_{1})+\delta u(c_{2})+...+\delta ^{T-1}u(c_{T}) 
	\]
	
	subject to%
	\[
	c_{1}+\frac{c_{2}}{1+r}+...+\frac{c_{T}}{(1+r)^{T-1}}\leq W. 
	\]
\end{quotation}

In this model, there are three motivations that influence the allocation of
consumption across periods: \textquotedblleft impatience\textquotedblright ,
\textquotedblleft future consumption is cheaper\textquotedblright , and
\textquotedblleft consumption smoothing\textquotedblright .

\textbf{(a)} For each of these motivations, (i) describe in words how the
motivation impacts the allocation of consumption across periods (i.e., how $%
c_{1}$ compares to $c_{2}$, $c_{3}$, and so forth), (ii) describe
mathematically how it enters the problem above, and (iii) describe
mathematically the special case of the problem above when the motivation is
absent.
\textbf{[Note: You should have a total of 9 responses for part a]}

\textbf{(b)} Suppose that $T=2$ and $u(c)=a+bc$. Derive the optimal
consumption path as a function of $\delta $ and $r$.

Hint: Use the substitution method in part \textbf{(b)}.

Hint: Your answer to part \textbf{(b)} might provide insight that helps you
to answer part \textbf{(a)}.

\end{document}
