%2multibyte Version: 5.50.0.2960 CodePage: 1251
%\renewcommand{\baselinestretch}{1.8}
%\input{tcilatex}
%\input{tcilatex}
%\usepackage{harvard}
%\input{tcilatex}
%\input{tcilatex}


\documentclass[12pt]{article}
%%%%%%%%%%%%%%%%%%%%%%%%%%%%%%%%%%%%%%%%%%%%%%%%%%%%%%%%%%%%%%%%%%%%%%%%%%%%%%%%%%%%%%%%%%%%%%%%%%%%%%%%%%%%%%%%%%%%%%%%%%%%%%%%%%%%%%%%%%%%%%%%%%%%%%%%%%%%%%%%%%%%%%%%%%%%%%%%%%%%%%%%%%%%%%%%%%%%%%%%%%%%%%%%%%%%%%%%%%%%%%%%%%%%%%%%%%%%%%%%%%%%%%%%%%%
\usepackage{amsfonts}
\usepackage{amssymb}
\usepackage{amsmath}
\usepackage{graphicx}
\usepackage{amsthm}
\usepackage{framed}
\usepackage{times}
\usepackage{mathptmx}
\usepackage{clrscode}
\usepackage{natbib}
\usepackage{booktabs,caption,fixltx2e}
\usepackage[flushleft]{threeparttable}
\usepackage{xpatch}
\usepackage{tabu}
\usepackage{setspace}
\usepackage[margin=1in]{geometry}

\setcounter{MaxMatrixCols}{10}
%TCIDATA{OutputFilter=LATEX.DLL}
%TCIDATA{Version=5.50.0.2960}
%TCIDATA{Codepage=1251}
%TCIDATA{<META NAME="SaveForMode" CONTENT="1">}
%TCIDATA{BibliographyScheme=Manual}
%TCIDATA{Created=Friday, July 11, 2003 08:16:50}
%TCIDATA{LastRevised=Tuesday, July 22, 2014 01:00:36}
%TCIDATA{<META NAME="GraphicsSave" CONTENT="32">}
%TCIDATA{Language=American English}

\makeatletter
\chardef\TPT@@@asteriskcatcode=\catcode`*
\catcode`*=11
\xpatchcmd{\threeparttable}
{\TPT@hookin{tabular}}
{\TPT@hookin{tabular}\TPT@hookin{tabu}}
{}{}
\catcode`*=\TPT@@@asteriskcatcode
\makeatother
\parskip 0cm
\parindent .4cm
\textwidth=6.5in
\oddsidemargin .0in
\topmargin -.5in
\textheight 9in
\doublespacing
\newtheorem{proposition}{Proposition}
\newtheorem{conjecture}{Conjecture}
\newtheorem{corollary}{Corollary}
\newtheorem{lemma}{Lemma}
\theoremstyle{definition}
\newtheorem{definition}{Definition}
\newtheorem{assumption}{Assumption}
\newtheorem{example}{Example}
\newtheorem{am}{Alternative Model}
\renewcommand{\proofname}{\bf Proof}
\renewcommand{\qedsymbol}{$\blacksquare$}

\newcommand{\bpi}{\boldsymbol{\pi}}
\newcommand{\he}{\hat{\eta}}
\newcommand{\etahat}{\hat{\eta}}
\newcommand{\hx}{\hat{x}}
\newcommand{\that}{\hat{\theta}}
\newcommand{\hv}{\hat{\theta}}
\newcommand{\hvl}{\hv_L}
\newcommand{\hvh}{\hv_H}
\newcommand{\vl}{\theta_L}
\newcommand{\vh}{\theta_H}
\newcommand{\hp}{\hat{p}}
%\newcommand{\bv}{\bar{v}}
\newcommand{\bv}{\theta}
\newcommand{\s}{\sigma}
\newcommand{\E}{\mathbb{E}}
\newcommand{\ind}{\mathbbm{1}}
\newcommand{\bigmid}{\;\bigg|\;}
\newcommand{\lbigmid}{\;\big|\;}
\newcommand{\cas}{\overset{\textnormal{a.s.}}{\longrightarrow}} 


\newcommand{\al}{\alpha}


%from P and A file

\newcommand{\hy}{\hat{y}}
\newcommand{\hh}{\hat{h}}
\newcommand{\hI}{\hat{I}}
\newcommand{\hF}{\hat{F}}
\newcommand{\hE}{\hat{E}}

\newcommand{\ah}{\hat{\theta}}
\newcommand{\at}{\tilde{\theta}}
\newcommand{\Fh}{\widehat{F}}
\newcommand{\eh}{\hat{e}}
\newcommand{\yh}{\hat{y}}
%\newcommand{\sh}{\widehat{\sigma}}
\newcommand{\sh}{\sigma}

\newcommand{\et}{\tilde{e}}
\newcommand{\zt}{\tilde{z}}
\newcommand{\Zt}{\widetilde{Z}}
\newcommand{\es}{e^*}
\newcommand{\xh}{\hat{x}}
\newcommand{\zh}{\hat{z}}
\newcommand{\Zh}{\widehat{Z}}
\newcommand{\e}{\epsilon}
\newcommand{\D}{\partial}
\newcommand{\Eh}{\widehat{\mathbb{E}}}


\raggedbottom

\begin{document}


\begin{center}
PRELIM \#1

\vspace*{.25in}

Economics 404\\[0pt]
Professor Ben Bushong\\[0pt]
Fall 2021

\vspace*{.25in}

Exam Date: Tuesday, October 5
\end{center}

\vspace*{.1in}
\singlespacing
You have approximately 80 minutes to complete this exam, and there are 140 points. Time should not be a binding constraint. I suggest that you allocate your time as if you had 70 minutes, so that you'll then have an extra 10 minutes to review your answers or to work on questions for which you needed more time. If you do so, then there are approximately twice as many points as there are minutes of exam time, and so, for instance, you should allocate approximately 6 minutes for a 12-point question and approximately 15 minutes for a 30-point question. On all questions, please clearly indicate what your final answers are by ``boxing” them, and recognize that multiple or vague answers will be graded with maximal skepticism. The exam is graded for correctness: when no explanations are asked for, you will get full credit without supplying them. (As always) I encourage complementing the algebra with intuition and logic at appropriate points so as to verify that your answers make sense, and to simplify your tasks.

\vspace*{.3in}

Be mellow. Don’t freak out. You are an intelligent and beautiful person. (Obviously the other members of this class are getting the same message. But in their case, I am just saying it to build confidence and security as they start the exam. In your case, I really mean it.)

{\centering \noindent For each question please \textbf{clearly identify} the question number. Again, please put a large BOX around each answer. \par
}

~

\noindent You must finish \textbf{before} 4:01 PM. 

\bigskip

\bigskip

\begin{center}
--------------------------------------------------------------------------------------------------------------

YOU MAY NOT BEGIN UNTIL YOU ARE TOLD TO BEGIN

--------------------------------------------------------------------------------------------------------------
\end{center}

\pagebreak
\doublespacing
\begin{center}
--------------------------------------------------------------------------------------------------------------

\textbf{START OF EXAM - PLEASE USE EXAM BOOK A} 

--------------------------------------------------------------------------------------------------------------
\end{center}

\bigskip

\underline{Question 1}: (12 points)

Rosalind and Anna are both expected utility maximizers, and they both have a
CRRA utility function $u(x)=x^{1-\rho }/(1-\rho )$. This is all you know about Rosalind and Anna.

If we observe that, when faced with the choice below, Rosalind chooses
Lottery A while Anna chooses Lottery B, what can we conclude---if anything---about who has a
larger $\rho $? (You should not need to do difficult math.) Briefly explain your answer and any assumptions you must make, if any. 

\begin{center}
Lottery A: $($ $600$ $,$ $\frac{1}{2}$ $;$ $300$ $,$ $\frac{1}{2}$ $)\qquad $%
vs.\qquad Lottery B: $($ $440$ $,$ $1$ $)$
\end{center}

\bigskip

\underline{Question 2}: (24 points)

Suppose that Martin owns a raffle ticket that yields a 1\% chance of
winning \$10000, a 10\% chance of winning \$100, and a 20\% chance of winning
\$10 (and otherwise he gets nothing). Suppose further that someone has
offered to buy the ticket for an amount $z$, and let $\bar{z}$ denote the
amount such that Martin is indifferent between keeping the raffle ticket
vs. selling it.

\textbf{(a)} Suppose that Martin is a risk-averse expected utility
maximizer. What can we conclude about his $\bar{z}$? 

\textbf{(b)} Suppose that Martin behaves according to prospect theory,
except that he has $\pi (p)=p$. What can we conclude about his $\bar{z}$?

\textbf{(c)} Suppose that Martin behaves according to prospect theory,
except that he has $\pi (p)=p$ and he has no diminishing sensitivity. What
can we conclude about his $\bar{z}$?

\vspace*{.5in}

\pagebreak

\begin{center}
\underline{\textbf{Part 1 (continued)}}
\end{center}

\vspace*{0.5in}

\underline{Question 3}: (16 points)

Suppose that Claire purchased an NFT of a cat meme in 2020 for \$1,300,000. (NFT stands for ``non-fungible token"; it is a way to purchase digital art and be the sole owner).  Shortly thereafter, the market value of this asset decreased to
\$1,000,000, and remained there for five months. In January 2021, Claire read
an article that described how meme NFTs (much like other meme assets) were expected by many analysts to increase in
value by 50\% over the next six months.

Now suppose that in June 2021, Claire is offered \$1,000,000 for her cat-meme NFT. If Claire experiences financial loss aversion, is she likely to
view selling the painting at \$1,000,000 as a gain or a loss? \textbf{In a few sentences, explain your
answer.}

Note: It could be that there is no single correct answer. If so, you should
discuss intelligently what the possible answers are.

\bigskip

\underline{Question 4}: (12 points)

According to the Koszegi-Rabin model, people's reference points are
determined by their recent expectations about outcomes. Suppose that, for an
endowment effect experiment, people's expectations --- and thus their
reference points --- are formed \textbf{prior to entering the lab or classroom} where the experiment ultimately takes place. According to the
Koszegi-Rabin model, would we expect such people to exhibit a larger or
smaller endowment effect than is predicted by the endowment effect model
that we studied in class? \textbf{In a few sentences, briefly explain your answer.}

\bigskip


\pagebreak

\begin{center}
	--------------------------------------------------------------------------------------------------------------
	
	\textbf{PART 2. PLEASE USE EXAM BOOK B} 
	
	--------------------------------------------------------------------------------------------------------------
\end{center}
\vspace*{.25in}

\underline{Question 5}: (30 points)

Suppose that Jeffrey has initial wealth \$10,000, but, before he consumes
it, he is subject to the following health risk (these events are mutually
exclusive):

\begin{center}
\begin{tabular}[t]{cc}
Required Medical Payment & Probability \\ 
\$1000 & 10\% \\ 
\$500 & 20\%%
\end{tabular}
\end{center}

An insurance agent offers Jeffrey the option to buy partial insurance,
wherein he can pay a premium $\alpha z$ to insure against proportion $\alpha 
$ of any medical payment that he needs to make (e.g., if he needs to make
the \$1000 payment, the insurance company would pay $\$1000\alpha $).
Jeffrey is restricted to choosing $\alpha \in \lbrack 0,1]$.

\textbf{(a)} Suppose Jeffrey is an expected utility maximizer with initial
wealth $w$ and utility function $u(x)$. Provide an equation that reflects
his utility as a function of $\alpha $ and $z$.

\textbf{(b)} Suppose further that Jeffrey has utility function $u(x)=x$.
Which $\alpha $ will he choose? Note: Your answer might
depend on $z$.

\textbf{(c)} Suppose instead that Jeffrey has initial wealth $w$ but he
evaluates gambles according to prospect theory with probability-weighting
function $\pi (p)$ and value function $v(x)$. Provide an equation that
reflects his prospective utility as a function of $\alpha $ and $z$.

\textbf{(d)} Suppose further that Jeffrey has probability-weighting function
is $\pi (p)=p$ and value function

\[
v(x)=\left\{ 
\begin{array}{ccc}
x &  & \text{if }x\geq 0 \\ 
\lambda x &  & \text{if }x\leq 0.%
\end{array}%
\right. 
\]

Which $\alpha $ will he choose? Note: Your answer might
depend on $z$.

\pagebreak

\begin{center}
\underline{\textbf{Exam (continued)}}
\end{center}

\underline{Question 6}: (30 points)

\bigskip 
 

Suppose that we observe the following preference (where $X>0$ or $X<0$):%
\[
\left( \text{ }\$0.1X\text{ },\text{ }1\text{ }\right) \succ (\text{ }\$0%
\text{ },\text{ }0.9\text{ };\text{ }\$X\text{ },\text{ }0.1\text{ }) 
\]

(a) Suppose that Skye is a risk-averse expected utility maximizer. For what
values of $X$ could Skye exhibit this preference?

(b) Suppose that Jane evaluates gambles according to prospect theory with $%
\pi (p)=p$ and a value function that has the three properties suggested by
Kahneman \& Tversky. For what values of $X$ could Jane exhibit this
preference?

(c) Suppose that Elizabeth evaluates gambles according to prospect theory
with a value function that has the three properties suggested by Kahneman \&
Tversky, and with a probability weighting function that has $\pi (0.1)>0.1$.
For what values of $X$ could Elizabeth exhibit this preference?

\bigskip

Suppose instead that we observe the following preference (where $Y>0$ or $Y<0
$):%
\[
(\text{ }\$0\text{ },\text{ }0.6\text{ };\text{ }\$Y\text{ },\text{ }0.4%
\text{ })\succ \left( \text{ }\$0.4Y\text{ },\text{ }1\text{ }\right) 
\]

(d) Suppose that Skye is a risk-averse expected utility maximizer. For what
values of $Y$ could Skye exhibit this preference?

(e) Suppose that Jane evaluates gambles according to prospect theory with $%
\pi (p)=p$ and a value function that has the three properties suggested by
Kahneman \& Tversky. For what values of $Y$ could Jane exhibit this
preference?

(f) Suppose that Elizabeth evaluates gambles according to prospect theory
with a value function that has the three properties suggested by Kahneman \&
Tversky, and with a probability weighting function that has $\pi (0.4)<0.4$.
For what values of $Y$ could Elizabeth exhibit this preference?

\vspace*{0.4in}

\pagebreak

\begin{center}
\underline{\textbf{Exam (continued)}}
\end{center}

\vspace*{0.4in}

\underline{Question 7}: (16 points)

Suppose that Iantha faces a choice between Lottery A and Lottery B:

\begin{center}
Lottery A: $($ $-100$ $,$ $\frac{1}{4}$ $;$ $0$ $,$ $\frac{1}{2}$ $;$ $400$ $%
,$ $\frac{1}{4}$ $)\qquad $vs.\qquad Lottery B: $($ $100$ $,$ $1$ $)$
\end{center}

Iantha behaves according to the Koszegi-Rabin model with $u(x | r)$ given by the following:%
\[
u (x | r)=\left\{ 
\begin{array}{cc}
x +  \eta (x - r) & \text{if }\quad x - r\geq 0 \\ 
x + \eta \lambda (x - r) & \text{if }\quad x-r < 0.%
\end{array}%
\right. 
\]

Suppose that Iantha has been expecting to face Lottery B.

\textbf{(a)} Derive Iantha's utility from Lottery A.

\textbf{(b)} Derive Iantha's utility from Lottery B.

\textbf{(c)} If $\eta =\frac{1}{2}$ and $\lambda =2$, which lottery will
Iantha choose?

\pagebreak

\noindent \underline{Super Secret Question.} \textbf{Do not attempt this question unless you have finished all other questions. Points will be awarded only for correct answers. (12 bonus points)} 
Suppose there are two goods, candy bars ($c$) and money ($m$). Paige has
initial income $I$, and she is deciding whether to buy 0, 1, or 2 candy bars
at a price of $p$ per candy bar. Paige's total utility is the sum of her
candy-bar utility and her money utility, and her intrinsic utilities for the
two goods are:%
\begin{eqnarray*}
	w_{c}(c) &\equiv &\left\{ 
	\begin{array}{cc}
		0 & \text{if }c=0 \\ 
		\theta _{1} & \text{if }c=1 \\ 
		\theta _{1}+\theta _{2} & \text{if }c=2%
	\end{array}%
	\right. \qquad \qquad \text{where }\theta _{1}>\theta _{2} \\
	&& \\
	\text{and }w_{m}(m) &\equiv &m
\end{eqnarray*}

\textbf{(a)} If Paige were a standard agent who only cares about her
intrinsic utilities, how would she behave as a function of the price $p$? In
other words, for what prices would she buy zero candy bars, for what prices
would she buy one candy bar, and for what prices would she buy two candy
bars?

\textbf{(b)} Now suppose that Paige behaves according to the Koszegi-Rabin
model. In other words, in addition to intrinsic utilities, she also cares
about gain-loss utility, where the gain-loss utility for each good is
derived from the universal gain-loss function:%
\[
\mu (z)=\left\{ 
\begin{array}{cc}
\eta z & \text{if }z\geq 0 \\ 
\eta \lambda z & \text{if }z\leq 0%
\end{array}%
\right. 
\]

If Paige expects to buy no candy bars, how would she behave as a function of the price $p$? In other words, for what prices would she buy zero candy bars, for what prices would she buy one candy bar, and for what prices would she buy two candy bars?

\begin{center}
--------------------------------------------------------------------------------------------------------------

\textbf{\underline{END OF EXAM} -- PLEASE ENSURE QUESTIONS ARE CLEARLY MARKED AND ANSWERS ARE BOXED}

--------------------------------------------------------------------------------------------------------------%

\end{center}

\end{document}
